\section{Entwicklungsumgebung}
\begin{frame}{Was ist eine Entwicklungsumgebung?}
	\begin{block}{IDE}
		Integrated Development Environment
	\end{block}
	\pause
	\begin{block}{Sammlung \emph{und Integration} von Werkzeugen}
		\begin{itemize}
			\item intelligenter Texteditor
			\item grafischer Debugger
			\item Autovervollständigung
			\item einfache Anbindung an den Compiler
			\item Hilfefunktion
			\item meist erweiterbar (z.B. git-Integration)
		\end{itemize}
	\end{block}
\end{frame}

\begin{frame}{Wofür?}
	\begin{description}
		\item[grafisch] IDEs sind üblicherweise grafisch, man kann daher viele Dinge über GUIs konfigurieren und anstoßen
		\item[integriert] es werden zahlreiche Werkzeuge miteinander vereint, z.B. ein Text-Editor mit einem Debugger
		\item[strukturiert] in der IDE wird zumeist auch die Struktur eines Projektes abgebildet (Ressourcen, source code usw.)
		\item[editor] die IDE kennt das gesamte Projekt und kann daher die Informationen aus allen Dateien nutzen (autocompletion, definition lookup etc.)
		\item[code analysis] durch ihre umfassendes Wissen kann die IDE auch den code analysieren, etwa um die Struktur von Klassen darstellen zu können
		\item[refactoring] das Umstrukturieren von code ist in gewissen Grenzen ebenfalls automatisiert möglich
	\end{description}
\end{frame}

\begin{frame}{Beispiele}
	\begin{itemize}
		\item Code::Blocks
		\item \emph{Eclipse}
		\item GNU Emacs
		\item KDevelop
		\item NetBeans
		\item Qt Creator
		\item proprietär: Visual Studio (Microsoft / Windows)
		\item proprietär: Xcode (Apple / OS~X, iOS)
		\item \dots
	\end{itemize}
\end{frame}