\section{Organisatorisches}


\subsection{Platzzahl}

\begin{frame}{Begrenzung der Platzzahl}
	\begin{block}{Warum eine Begrenzung?}
		\begin{itemize}
			\item Raumgröße
			\item Betreuung, Gruppengröße
		\end{itemize}
	\end{block}
	\ \\
	\pause
	\ \\
	\begin{block}{Platzzahl}
		\begin{itemize}
			\item Max. 24 Plätze, Betreuer zusätzlich
			\item Warteschlange (wie bei Sprachkurs)
			\item Online-Teilnahme möglich
			\begin{itemize}
				\item Die Vorträge werden (versuchsweise) als Screencast aufgezeichnet
				\item Fragen können über die Mailinglist gestellt werden
				\item \dots oder über unseren IRC-Channel
			\end{itemize}
		\end{itemize}
	\end{block}
\end{frame}

\begin{frame}[fragile]{Wie bekomme ich einen Platz?}
	Nach folgenden Kriterien werden die Plätze verteilt:
	\begin{enumerate}
		\item Anwesenheit heute (27.4.) oder Entschuldigung
		\item Erforderliche Programmierkenntnisse
		\begin{itemize}
			\item \enquote{Programmieren für Physiker} als Referenz
			\item Codebeispiel zum ersten Termin in \verb|example_apps/easy.cpp| als Minimalanforderung
		\end{itemize}
		\item Aktive Teilnahme und Interesse (der \emph{Willen}, teilzunehmen und das durchzuziehen)
		\item first-come, first-served (Wer zuerst kommt, mahlt zuerst.)
	\end{enumerate}
	
	Nachrücken ist möglich!
\end{frame}



\subsection{Organisatorisches}

\begin{frame}{Kontakt, Kooperation}
	\begin{description}
		\item[Mailing-List] cpp-workshop@lists.kit.edu
		\item[IRC] \#kit-cpp-workshop auf euirc.net
		\item[GitHub]	\url{www.github.com}, Organization kit-cpp-workshop
	\end{description}
	\ \\
	
	Namen und E-Mail-Adressen:
	\begin{table}
		\begin{tabular}{l|l}
			Christian Käser	&	Christian.Kaeser@student.kit.edu	\\
			\hline
			Markus Jung		&	Markus.Jung@stud.uni-karlsruhe.de	\\
			\hline
			Matthias Blaicher	&	matthias@blaicher.com	\\
			\hline
			Robert Schneider	&	Robert.Schneider3@student.kit.edu	\\
			\hline
			Sven Brauch	&	SvenBrauch@googlemail.com	\\
		\end{tabular}
	\end{table}
\end{frame}

\begin{frame}{Der Workshop}
	\begin{block}{Termin}
		\begin{itemize}
			\item jeden Freitag in der Vorlesungszeit
			\item 15:45-17:15
			\item Raum: 2-0 (Physik-Hochhaus)
			\item evtl. Zusatztermine nach Absprache (z.B. cmake)
		\end{itemize}
	\end{block}
	\pause
	\begin{block}{Ziele}
		\begin{itemize}
			\item Erfahrungsaustausch!
			\item Hilfe zur Selbsthilfe / Anhaltspunkte zum Weiterlernen
			\item Tiefergehenderer Einblick in die Sprache
			\item Verschiedene Gebiete von C++ (\enquote{Schweizer Taschenmesser})
			\item Kooperation, Team-Programmierung
			\item Software-Design, programming idioms, Fehler vermeiden usw.
		\end{itemize}
	\end{block}
\end{frame}

\begin{frame}{Struktur eines Workshops}
	\begin{block}{Themen}
		\begin{itemize}
			\item Richten sich nach den Interessen der Teilnehmer!
			\item Grundlegende Methodiken, Werkzeuge, ...
			\item Objektorientierung
			\item Algorithmen \& Datenstrukturen
			\item Verwendung von Bibliotheken (statisch, dynamisch gelinkt)
		\end{itemize}
	\end{block}
	\pause
	\begin{block}{Aufteilung}
		\begin{itemize}
			\item Theoretischer Teil, heute: Organisatorisches, git, eclipse
			\item Praktischer Teil: pair programming (2er-Teams) IDE + hello-world
			\item Übungen: im praktischen Teil, zu Hause fertig. Vorstellung einiger Lösungen in darauffolgender Woche
		\end{itemize}
	\end{block}
\end{frame}

