\section{Praxis}
\begin{frame}[fragile]{Und jetzt: Praxis!}
	\scriptsize
	\begin{block}{Hello World, Test unseres Workflows}
		\begin{enumerate}
			\item	Forke das Repository und klone es auf deinen Rechner:
		\url{https://github.com/kit-cpp-workshop/workshop-ss12-01}
			\item Erstelle ein eclipse-Projekt unter \emph{task01} {\tiny(damit ist \verb|task01.cpp| automatisch im Projekt enthalten)}.
			\item Ersetze in \emph{task01} den Text \enquote{Hello World!} durch einen Text mit deinem Namen, builde das Projekt und überprüfe das Resultat.
			\item Erstelle einen commit, pushe (Git) und erstelle einen pull request (auf github), in welchem der Namen deines Betreuers steht.
		\end{enumerate}
	\end{block}
	
	\begin{block}{Eclipse / cpp: Fibonacci-Zahlen}
		Lies eine Zahl n vom Benutzer ein und gib die ersten n Fibonacci-Zahlen durch Zeilenumbrüche getrennt auf der Kommandozeile aus. Nutze die Vorlage in \emph{task02} oder zumindest den Ordner \emph{task02}.
	\end{block}
	
	\begin{block}{Bonusaufgabe: Project Euler}
		Falls du früher fertig bist oder einfach nur Lust auf mehr hast, schau dich mal bei \url{http://projecteuler.net/} um und such dir eine Aufgabe aus. Lade sie genau wie die beiden anderen Aufgaben hoch.
	\end{block}
\end{frame}
